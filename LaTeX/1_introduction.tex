%% This is an example first chapter.  You should put chapter/appendix that you
%% write into a separate file, and add a line \include{yourfilename} to
%% main.tex, where `yourfilename.tex' is the name of the chapter/appendix file.
%% You can process specific files by typing their names in at the 
%% \files=
%% prompt when you run the file main.tex through LaTeX.
\chapter{Introduction}

\section{The Aerocene project}

\section{Acceleration from geopotential height} \label{sec:acceleration}
In a dynamic isobaric trajectory model, the acceleration of a parcel of air at a given pressure level can be determined from its velocity, latitude, and geopotential height.
The geopotential height $Z_g$ of a surface of pressure $p$ above mean sea level is 

\begin{align}
    Z_g(p) = R \int_p^{p_s} \frac{T}{g} \frac{dp}{p},
\end{align}

where $R$ is the gas constant, $T$ is temperature, $p_s$ is surface pressure, $g$ is acceleration due to gravity at mean sea level, and $Z_g(p_s)$ is set to 0. 
In the troposphere, the difference between geopotential height and actual height is negligible \cite{marshall_atmosphere_2008}. 

The geopotential $\Phi$ is the potential energy of the Earth's gravitational field at a height $h$:

\begin{align}
    \Phi (h) = g Z_g (h) \label{eq:geopotential}.
\end{align}

The full equation for acceleration of an air parcel in the atmosphere is

\begin{align}
    \frac{D \vec{u}}{Dt} + \frac{1}{\rho} \nabla p + \nabla \Phi + f \hat{z} \times \vec{u} = \mathcal{F} \label{eq:full}
\end{align}

where $\vec{u}$ is the wind velocity vector, $\rho$ is the density of air, $f$ is the Coriolis parameter, and $\mathcal{F}$ is the force of friction per unit mass \cite{marshall_atmosphere_2008}.
To implement this equation in the dynamic model, it must be written in terms of the zonal, meridional, and vertical wind components $u$, $v$ and $w$.
One assumption of the model is that friction is negligible at the studied pressure level, so $\mathcal{F}$ is taken to be zero.

\begin{align}
    \frac{Du}{Dt} + \frac{1}{\rho} \frac{\partial p}{\partial x} - fv &= 0 \label{eq:u_component} \\
    \frac{Dv}{Dt} + \frac{1}{\rho} \frac{\partial p}{\partial y} - fu &= 0 \label{eq:v_component} \\
    \frac{Dw}{Dt} + \frac{1}{\rho} \frac{\partial p}{\partial z} + g &= 0. \label{eq:vertical}
\end{align}
 
The derivative of the geopotential is zero in the horizontal directions, and equal to $g$ in the vertical following from Equation \ref{eq:geopotential}.
Due to the isobaric assumption, the vertical velocity is zero, so Equation \ref{eq:vertical} becomes an expression of hydrostatic equilibrium:

\begin{align}
    \frac{1}{\rho} \frac{\partial p}{\partial z} + g &= 0, \text{ or} \\
    \frac{\partial p}{\partial z} &= - \rho g. \label{eq:hydrostatic}
\end{align}

Pressure can be defined as a function of the horizontal position and the geopotential height at that position at a given time:

\begin{align}
    p(x, y, Z_g(x, y, t), t) = p_0
\end{align}

where $p_0$ is the studied pressure level.
The partial derivatives of pressure with respect to $x$ and $y$ are

\begin{align}
    \frac{\partial p}{\partial x} + \frac{\partial p}{\partial Z_g} \frac{\partial Z_g}{\partial x} &= 0 \\
    \frac{\partial p}{\partial y} + \frac{\partial p}{\partial Z_g} \frac{\partial Z_g}{\partial y} &= 0.
\end{align}

Using the hydrostatic relationship in Equation \ref{eq:hydrostatic}, these become

\begin{align}
    \frac{\partial p}{\partial x} &= \rho g \frac{\partial Z_g}{\partial x} \\
    \frac{\partial p}{\partial y} &= \rho g \frac{\partial Z_g}{\partial y}
\end{align}

which can be substituted into Equations \ref{eq:u_component} and \ref{eq:v_component} to yield

\begin{align}
    \frac{Du}{Dv} &= fv - g \frac{\partial Z_g}{\partial x} \label{eq:uvelocity} \\
    \frac{Dv}{Dt} &= - fu - f \frac{\partial Z_g}{\partial y}. \label{eq:vvelocity}
\end{align}

These equations are implemented in the dynamic model to determine the acceleration of air parcels over time.
