%% This is an example first chapter.  You should put chapter/appendix that you
%% write into a separate file, and add a line \include{yourfilename} to
%% main.tex, where `yourfilename.tex' is the name of the chapter/appendix file.
%% You can process specific files by typing their names in at the 
%% \files=
%% prompt when you run the file main.tex through LaTeX.
\chapter{Introduction}

\section{The Aerocene Project}

\section{Acceleration from Geopotential Height} \label{sec:acceleration}
In an isobaric trajectory model, the acceleration of a parcel of air at a given pressure level can be determined from its velocity, latitude, and geopotential height.
The geopotential height $Z_g$ of a surface of pressure $p$ above mean sea level is 

\begin{align}
    Z_g(p) = R \int_p^{p_s} \frac{T}{g} \frac{dp}{p},
\end{align}

where $R$ is the gas constant, $T$ is temperature, $p_s$ is surface pressure, $g$ is acceleration due to gravity at mean sea level, and $Z_g(p_s)$ is set to 0. 
In the troposphere, the difference between geopotential height and actual height is negligible \cite{marshall_atmosphere_2008}. 

The geopotential $\Phi$ is the potential energy of the Earth's gravitational field at a height $h$:

\begin{align}
    \Phi (h) = g Z_g (h) \label{eq:geopotential}
\end{align}

The full equation for acceleration of an air parcel in the atmosphere is

\begin{align}
    \frac{D \vec{u}}{Dt} + \frac{1}{\rho} \nabla p + \nabla \Phi + f \hat{z} \times \vec{u} = \mathcal{F} \label{eq:full}
\end{align}

where $\vec{u}$ is the wind velocity vector, $\rho$ is the density of air, $f$ is the Coriolis parameter, and $\mathcal{F}$ is the force of friction \cite{marshall_atmosphere_2008}. 
In an isobaric trajectory model, the gradient of pressure is zero, and the force of friction is assumed to be negligible at the studied pressure level.
Using these assumptions and the relation in \ref{eq:geopotential}, equation \ref{eq:full} becomes

\begin{align}
    \frac{D \vec{u}}{Dt} + \nabla \Phi + f \hat{z} \times \vec{u} &= 0 \\
    - g \nabla Z_g - f \hat{z} \times \vec{u} &= \frac{D \vec{u}}{Dt}
\end{align}

of which the horizontal components are

\begin{align}
    \frac{Du}{Dt} &= fv - g \frac{\partial Z_g}{\partial x} \label{eq:uvelocity} \\
    \frac{Dv}{Dt} &= -fu - g \frac{\partial Z_g}{\partial y}. \label{eq:vvelocity}
\end{align}

%% To include in intro
% Geopotential height explanation
% Advection equations use geopotential height gradient
% Factors that lead to ageostrophic flow?
