%% This is an example first chapter.  You should put chapter/appendix that you
%% write into a separate file, and add a line \include{yourfilename} to
%% main.tex, where `yourfilename.tex' is the name of the chapter/appendix file.
%% You can process specific files by typing their names in at the 
%% \files=
%% prompt when you run the file main.tex through LaTeX.
\chapter{Methods}

\section{Interpolation}

Both the kinematic and dynamic models require an interpolation scheme to produce values for atmospheric variables in between the gridded values provided by GFS. 
Linear interpolation is the standard choice for trajectory models \cite{bowman_input_2013}. 
For both models, linear interpolation was used in three dimensions (latitude, longitude, and time). 
In the kinematic model, $u$ and $v$ components of wind speed were interpolated, while in the dynamic model, geopotential height was interpolated.

\section{Integration Scheme}
The numerical scheme chosen was a second-order Runge-Kutta method with a long track record in trajectory modeling \cite{petterssen_weather_1940}.
The velocity at a given timestep is taken to be the average of the velocity at the initial position and the velocity at the first-guess position after one timestep.

The first guess position $\vec{P}' (t + \Delta t)$ is 
\begin{align}
\vec{P}' (t + \Delta t) = \vec{P}(t) + \vec{V} (\vec{P}, t) \Delta t
\end{align}

and the final position $\vec{P} (t + \Delta t)$ is
\begin{align}
\vec{P} (t + \Delta t) = \vec{P} (t) + \frac{1}{2} \left [ \vec{V} (\vec{P}, t) + \vec{V} (\vec{P'}, t + \Delta t) \right ] \Delta t
\end{align}

where $\vec{P}$ is a position vector with latitude and longitude components, and $\vec{V}$ a velocity vector with $u$ and $v$ wind speeds \cite{draxler_description_1997}.
This integration method is used by HYSPLIT and a number of other trajectory models, including FLEXPART, LAGRANTO, and STILT \cite{stein_noaas_2015} \cite{bowman_input_2013}. 
For trajectories calculated from interpolated gridded wind velocities, higher order integration schemes do not add precision \cite{draxler_description_1997}.  

\section{Timestep}

The timestep for integration was three minutes, with the timestep throughout the trajectory. 
To save computation, HYSPLIT uses a dynamic timestep, varying from one minute to one hour, computed to satisfy

\begin{align}
U_{max} [\text{grid-units min}^{-1}] \Delta t [\text{min}] < 0.75 [\text{grid-units}] 
\end{align}

\cite{draxler_description_1997}. This ensures that the parcel does not blow past any grid squares during a single timestep, which maximizes the accuracy of the calculation. 
%Check my trajectories to see if the U_max dt < 0.75 relation is always satisfied. If it isn't, consider reducing the timestep.